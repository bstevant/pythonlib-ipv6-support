\section{Pour conclure}

\frame{
	\frametitle{IPv6 : les point de vigilance}
	L'API socket facilite la vie par l'uniformisation
	
	$\implies$ pas besoin d'être masochiste !
	
	
	
	Par contre il y a d'autres points de vigilance :
	\begin{itemize}
		\item Entrée utilisateur des adresses 
		\item Manipulation d'adresses
		\item Stockage d'adresses dans des BDD
		\item $<$\small{insert your next bug here}$>$
	\end{itemize}
}


\pgfdeclareimage[interpolate=true,height=7cm]{github_ghbn}{./images/github_ghbn.png}
\frame{
	\frametitle{Il y a encore du boulot}
	\begin{tikzpicture}
		\clip (0, 0) rectangle (12,7);
		\pgfputat{\pgfxy(0,0)}{\pgfbox[left,base]{\pgfuseimage{github_ghbn}}}
	\end{tikzpicture}
}

\pgfdeclareimage[interpolate=true,height=7cm]{github_gai}{./images/github_gai.png}
\frame{
	\frametitle{Il y a encore du boulot}
	\begin{tikzpicture}
		\clip (0, 0) rectangle (12,7);
		\pgfputat{\pgfxy(0,0)}{\pgfbox[left,base]{\pgfuseimage{github_gai}}}
	\end{tikzpicture}
}

\frame{
	\frametitle{Pour aller plus loin}
	\begin{itemize}
		\item<2-> Le Blog de Stéphane Bortzmeyer
		\begin{itemize}
			\item \url{http://www.bortzmeyer.org/bindv6only.html}
			\item \url{http://www.bortzmeyer.org/resolution-noms-api.html}
		\end{itemize}
		\item<3-> Le MOOC Objectif IPv6
		\begin{itemize}
			\item \url{http://fun-mooc.fr}
			\item Supports en cours de publication (libre!)
		\end{itemize}
		\item<4-> L'association G6
		\begin{itemize}
			\item \url{http://g6.asso.fr}
			\item Listes de diffusions pour échanger sur IPv6
		\end{itemize}
		
	\end{itemize}

}
