\section{IPv6 et Python}

\frame{
	\frametitle{L'héritage socket}
	Le support réseau de Python repose sur l'API \texttt{socket}
	\begin{itemize}
		\item Héritage de l'API C
		\item Utilisée par Un bon nombre (toutes?) des librairies réseaux
		\begin{itemize}
			\item socketserver
			\item urllib
			\item asyncio
			\item TwistedMatrix
			\item HTTPLib2
			\item etc.
		\end{itemize}
	\end{itemize}
	Cette API est conforme au RFC 3493:
	\begin{itemize}
		\item La socket ouverte en IPv6
		\item Les connexions IPv4 sont \textit{mappée} en connexions IPv6
	\end{itemize}
}

\begin{frame}[fragile]{}
	\frametitle{Un rappel sur la socket}
	Une socket est identifiée par 5 paramètres:
	\begin{itemize}
		\item Le protocole de transport (TCP/UDP)
		\item Adresse de l'hôte expéditeur
		\item Port de l'application origine
		\item Adresse de l'hôte destinataire
		\item Port de l'application cible
	\end{itemize}
	
	Sous Linux :
	\begin{verbatim}
% netstat -taun
Connexions Internet actives (serveurs et établies)
Proto Recv-Q Send-Q Adresse locale          Adresse distante        Etat      
tcp        0      1 10.51.0.215:43742       10.35.1.191:631         ESTABLISHED  
	\end{verbatim}
\end{frame}

\begin{frame}[fragile]{}
	\frametitle{Un petit serveur IPv4}
    \begin{minipage}{0.5\textwidth}
      \inputminted[mathescape,
        linenos,
        fontsize=\scriptsize,
        framesep=2mm]{python}{../echo_server_v4.py}
    \end{minipage}
	
\end{frame}


\begin{frame}[fragile]{}
	\frametitle{Testons !}
	
	\begin{verbatim}
% python ./echo_server_v4.py &
% nc 127.0.0.1 50007
Connected by ('127.0.0.1', 53112)
kl
kl
^Z
[2]  + 88737 suspended  nc 127.0.0.1 50007
 % netstat -tan | grep 50007
tcp4       0      0  127.0.0.1.50007        127.0.0.1.53112 
tcp4       0      0  127.0.0.1.53112        127.0.0.1.50007 
tcp4       0      0  *.50007                *.*             
%
	\end{verbatim}

\end{frame}

\begin{frame}[fragile]{}
	\frametitle{Une IPv6-fication naïve}
    \begin{minipage}{0.5\textwidth}
      \inputminted[mathescape,
        fontsize=\scriptsize,
        framesep=2mm]{diff}{../echo_server_v4_v6.diff}
    \end{minipage}
\end{frame}


\begin{frame}[fragile]{}
	\frametitle{Une IPv6-fication naïve}
    \begin{minipage}{0.5\textwidth}
      \inputminted[mathescape,
        linenos,
        fontsize=\scriptsize,
        framesep=2mm]{python}{../echo_server_v6_naif.py}
    \end{minipage}
\end{frame}



\begin{frame}[fragile]{}
	\frametitle{Testons !}
	
	\begin{verbatim}[fontsize=\tiny]
% python ./echo_server_v6_naif.py &
% nc ::1 50007 
Connected by ('::1', 53224, 0, 0)
dz
dz
^Z
[2]  + 88997 suspended  nc ::1 50007
% netstat -tan | grep 50007
tcp6       0      0  ::1.50007              ::1.53224
tcp6       0      0  ::1.53224              ::1.50007
tcp46      0      0  *.50007                *.*             
% nc 127.0.0.1 50007
Connected by ('::ffff:127.0.0.1', 53242, 0, 0)
lkjlkj
lkjlkj
^Z
[2]  + 89045 suspended  nc 127.0.0.1 50007
% netstat -tan | grep 50007
tcp6       0      0  ::1.50007              ::1.53224       
tcp6       0      0  ::1.53224              ::1.50007       
tcp4       0      0  127.0.0.1.50007        127.0.0.1.53242 
tcp4       0      0  127.0.0.1.53242        127.0.0.1.50007 
tcp46      0      0  *.50007                *.*             
	\end{verbatim}

\end{frame}

\frame{
	\frametitle{Pour résumer, côté serveur}
	L'API socket IPv6 apporte la compatibilité ascendante avec IPv4
	\begin{itemize}
		\item Les adresses IPv4 mappée représentent les hôtes IPv4
		\item Les connexions IPv4 sont vues comme des connexion IPv6
		\item L'application n'a plus à différencier la famille d'adresse
	\end{itemize}
	Attention, cette glue n'est disponible que sur les OS compatibles IPv6
	\begin{itemize}
		\item Désolé pas pour MS-DOS, BeOS, etc.
	\end{itemize}
}

%\frame{
%	\frametitle{Un dernier détail}
%	Attention au paramètre \texttt{HOST} donné à la fonction socket
%	\begin{itemize}
%		\item
%	\end{itemize}
%}

\frame{
	\frametitle{Côté client\ldots}
	Le client doit déterminer : 
	\begin{itemize}
		\item A quelle adresse envoyer sa requête
		\begin{itemize}
			\item<2->soit l'utilisateur fixe cette adresse
			\item<3->soit elle est à résoudre à partir d'un nom
		\end{itemize}
		\item A quelle adresse le serveur enverra la réponse
		\begin{itemize}
			\item<4->Sélection de l'adresse source
		\end{itemize}
	\end{itemize}
}

\frame{
	\frametitle{Résoudre un nom en adresse}
	C'est la fonction du DNS, appelé par la fonction \texttt{socket.getaddrinfo}\footnote{\texttt{gethostbyname} à proscrire car non compatible IPv6 !}
	\begin{itemize}
		\item Résout un nom en \emph{plusieurs} adresses \emph{IPv4 ou IPv6}
		\item Peut aussi rendre au choix des adresses IPv4-mappées (uniformisation!)
		\item Bonus: uniformisation des adresses fixées par l'utilisateur
	\end{itemize}
}

\begin{frame}[fragile]{}
	\frametitle{Un client IPv4}
    \begin{minipage}{0.5\textwidth}
      \inputminted[mathescape,
        linenos,
        fontsize=\scriptsize,
        framesep=2mm]{python}{../echo_client_v4.py}
    \end{minipage}
	
\end{frame}

\begin{frame}[fragile]{}
	\frametitle{Une IPv6-fication naïve}
    \begin{minipage}{0.5\textwidth}
      \inputminted[mathescape,
        fontsize=\scriptsize,
        framesep=2mm]{diff}{../echo_client_v4_v6.diff}
    \end{minipage}
\end{frame}


\begin{frame}[fragile]{}
	\frametitle{Une IPv6-fication naïve}
    \begin{minipage}{\textwidth}
      \inputminted[mathescape,
        linenos,
        fontsize=\scriptsize,
        framesep=2mm]{python}{../echo_client_v6_naif.py}
		Non satisfaisant: une seule adresse testée, pas d'uniformisation
    \end{minipage}
\end{frame}

% Bypass
\begin{frame}<presentation:0>[fragile]{}
	\frametitle{Une IPv6-fication plus \emph{smart}}
    \begin{minipage}{0.5\textwidth}
      \inputminted[mathescape,
        fontsize=\scriptsize,
        framesep=2mm]{diff}{../echo_client_v6_naif_v6.diff}
    \end{minipage}
\end{frame}


\begin{frame}[fragile]{}
	\frametitle{Une IPv6-fication plus \emph{smart}}
    \begin{minipage}{0.5\textwidth}
      \inputminted[mathescape,
        linenos,
        fontsize=\scriptsize,
        framesep=2mm]{python}{../echo_client_v6.py}
    \end{minipage}
\end{frame}


\frame{
	\frametitle{Sélection de l'adresse source}
	Ce n'est pas (normalement) du ressort du programme.
	
	Une interface peut avoir : 
	\begin{itemize}
		\item des adresses de familles différentes
		\item plusieurs adresses d'une même famille (notamment IPv6)
	\end{itemize}
	
	Laquelle choisir ?
	
	Certains cas sont délicats, mais il y a une RFC pour ça \url{http://www.bortzmeyer.org/6724.html}
}

\begin{frame}[fragile]{}
	\frametitle{Retour sur le serveur}
    \begin{minipage}{0.5\textwidth}
      \inputminted[mathescape,
        fontsize=\tiny,
        framesep=2mm]{diff}{../echo_server_v6_naif_v6.diff}
    \end{minipage}
\end{frame}


\begin{frame}[fragile]{}
	\frametitle{Un serveur IPv6 \emph{smart}}
    \begin{minipage}{\textwidth}
      \inputminted[mathescape,
        linenos,
        fontsize=\tiny,
        framesep=2mm]{python}{../echo_server_v6.py}
    \end{minipage}
\end{frame}

%\frame{
%	\frametitle{}
%	\begin{itemize}
%		\item
%	\end{itemize}
%}

